\documentclass[line,11pt,a4paper]{resume}

\usepackage{graphicx}
\usepackage[ngerman]{babel}
\usepackage[pdfborder={0 0 0}]{hyperref}
\usepackage{multicol}
\usepackage{fontawesome}

\newcommand{\mail}[1]{\textsl{\href{mailto:#1}{#1}}}

\addtolength{\topskip}{8mm}

\begin{document}
\name{\LARGE Dr.~Michael~A.~Wild}
\begin{resume}
\vspace{-4mm}
\-\hspace{-12.3mm}\begin{minipage}{15cm}
\href{https://linkedin.com/in/wildmichael}{\faLinkedinSquare/in/wildmichael}\quad
\href{https://xing.com/profile/Michael_Wild4/cv}{\faXingSquare/profile/Michael\_Wild4/cv}\quad
\href{https://github.com/wildmichael}{\faGithubSquare/wildmichael}
\end{minipage}


\begin{multicols}{3}

\section{\mysidestyle Kontaktdaten}\vspace{0.9mm}
Stockerenstrasse 29 \\
3065 Bolligen \\
Schweiz \\
+41 78 405 31 34 \\
\mail{mwild@bluewin.ch}\\

\columnbreak

\section{\mysidestyle Zur Person}\vspace{2mm}

\begin{tabular}{@{}ll}
\textsl{Geburtstag} & 1979-12-17 \\
\textsl{Nationalität}   & Schweizer
\end{tabular}

\columnbreak
\vspace*{-9mm}\hfill\includegraphics[width=30mm]{mwild}

\end{multicols}

\section{\mysidestyle Berufserfahrung}\vspace{2mm}

\textbf{Noser Engineering}\\\vspace{1mm}%
\textsl{Senior Development Engineer (Bern)}
  \hfill \textbf{August 2020 -- Heute}\\
\vspace{-4mm}%
\begin{list2}
  \item Architektur, Performance Analyse und Entwicklung für das
    Paket-Logistik-Planungstool der Schweizerischen Post.
  \item Requirements Engineering, Architektur und Design für ein
    spezialisiertes CRM und Auftragsabwicklungstool der Kommunikations- und
    Übersetzungsabteilung der Schweizerischen Post.
  \item Entwicklung einer Web-Applikation zur zentralen Konfiguration der
    internen Telefonie (Cisco Unified Communications Manager, Microsoft
    Skype, Mobiltelefonie) für die Schweizerische Post und deren Einbindung
    in diverse Umsysteme, wie Verrechnung, Microsoft Active Directory
    und dem CMDB System (C\#, MS~SQL~Server, ASP.NET~Core, Entity~Framework~Core,
    AutoMapper, Angular, RxJS).
  \item Analysieren von WTO-Ausschreibungen hinsichtlich technischer Aspekte
    und Risiken.
\end{list2}

\textbf{ANDRITZ Hydro}\\\vspace{1mm}%
\textsl{Head of Turbine Digitalization (Linz)}
  \hfill \textbf{Januar 2019 -- Juli 2020}\\
\vspace{-4mm}%
\begin{list2}
  \item Leitung eines Teams aus sieben Entwicklern. Rekrutierung von drei
    internationalen Entwicklern (Vietnam, Brasilien). Verantwortung für Budget
    und Reporting.
  \item Entwicklung und Wartung verschiedener Softwarepakete, die von der F\&E,
    dem technischen Verkauf und der Konstruktionsabteilung verwendet werden
    (C\#, WinForms, MS~SQL~Server, ASP.NET~WebApi, C++, MariaDB, Qt, Python).
    Agiles Projektmanagement, Continuous Integration und detaillierte
    Dokumentation stellen eine zeitgerechte Fertigstellung der Funktionalität
    und hohe Qualitätsstandards sicher. (Git/Bitbucket, Jira, Jenkins, Azure
    DevOps, Confluence).
  \item Umsetzung einer Datenbank für die parametrische CAD-Konstruktion.
    Integration in bestehende Systeme und Berechnungswerkzeuge und Schaffung
    digitaler Datenströme (.NET~Core WebApi, Angular, Blazor, Azure Active
    Directory, API Gateway Pattern).
  \item Enge Einbindung mehrerer bestehender Lösungen, Erstellen von digitalen
    Datenflüssen um die manuelle Datenerfassung zu reduzieren (C++, WebApi,
    Python).
\end{list2}

\textsl{Entwicklungsingenieur CAE Werkzeuge (Linz)}
  \hfill \textbf{August 2016 -- Dezember 2018}\\
\vspace{-4mm}%
\begin{list2}
  \item Agile Master und Systemarchitekt für die Industrial IoT Lösung METRIS
    DiOMera zur Überwachung und Optimierung von Wasserkraftanlagen
    (MS~SQL~Server und Analysis Services, Entity~Framework, InfluxDB, WebApi,
    SignalR, Python, Pandas, scikit-learn, Docker, Ansible, Azure).

  \item Anwendung von Machine Learning und statistischer Analyse für die
    Betriebsoptimierung von Wasserkraftanlagen und vorausschauende Wartung.
    Vorschlagen von Analysemethoden, Einbringung des notwendigen
    Domänenwissens, Entwicklung von Prototyp-Algorithmen (Python, Jupyter
    Notebooks, Pandas, Power~Query, Power~BI).
\end{list2}

\textsl{Entwicklungsingenieur CAE Werkzeuge (Zürich und Vevey)}
  \hfill \textbf{Mai 2013 -- Dezember 2018}\\
\vspace{-4mm}%
\begin{list2}
  \item Leiter einer internationalen Arbeitsgruppe zur Erstellung
    einer Harmonisierungsstrategie für die Hochleistungsrechner zwischen den
    verschiedenen ANDRITZ Hydro F\&E Standorten.

  \item Unter anderem Entwickler und Projektleiter von internen
    Engineering-Tools; Kontaktperson zur Patentabteilung; Scrum Master in
    mehreren erfolgreichen Projekten; Architekt und Designer mehrerer
    Microservices (C\#, WPF, WinForms, WebApi, MS~SQL~Server,
    Entity~Framework).
\end{list2}

\textbf{Basler Kantonalbank}\\\vspace{1mm}%
\textsl{Financial Engineer (Basel)}
  \hfill \textbf{Januar 2013 -- April 2013}\\
\vspace{-4mm}%
\begin{list2}
  \item Entwicklung und Wartung der Programme zur Verarbeitung und
    Zusammenführung von Daten der unterschiedlichen Handelssystem in die
    Kernbankensoftware (C, Perl, Python, IBM MQ, Sybase, Oracle DB, Windows, HP
    UX, IBM AIX, Solaris, RHEL).
\end{list2}

\textbf{Institut für Fluiddynamik}, ETH Zürich\\\vspace{1mm}%
\textsl{Wissenschaftlicher Mitarbeiter (Zürich)}
  \hfill \textbf{Juni 2006 -- Dezember 2012}\\
\vspace{-4mm}%
\begin{list2}
  \item Entwickler eines MPI-parallelen Lösers mit hybridem Ansatz aus
    der traditionellen Finite-Volumen-Methode und der Transportgleichung für
    die Verbundswahrscheinlichkeitsdichtefunktion mittels einer
    Partikel-Monte-Carlo Methode. Die Software basiert auf der Open-Source
    Bibliothek OpenFOAM (C++, Python, Bash, Linux, ParaView).

  \item Erfinder mehrerer neuartiger Algorithmen, welche die Stabilität und
    Effizienz der hybriden FV-JPDF Simulationsmethode verbesserten.
\end{list2}

\section{\mysidestyle Ausbildung}\vspace{2mm}

\textbf{Berner Fachhochschule} \hfill \textbf{September 2020}\\
\vspace{1mm}%
\textsl{Zertifizierung Hermes Foundation}\\
\vspace{-1mm}%
%
\textbf{Scrum.org -- Zühlke Academy} \hfill \textbf{November 2013}\\
\vspace{1mm}%
\textsl{Professional Scrum Master (PSM I)}\\
\vspace{-1mm}%
%
\textbf{Institut für Fluid Dynamik -- ETH Zürich}
  \hfill \textbf{Juni 2006 -- Dezember 2012}\\
\vspace{1mm}%
\textsl{Dr\ sc\ ETH}\\
\vspace{-1mm}%
%
\textbf{University of Cambridge -- ESOL Examinations, Switzerland}
  \hfill \textbf{Juni 2007}\\
\vspace{1mm}%
\textsl{Certificate of Proficiency in English (CPE)}\\
\vspace{-1mm}%
%
\textbf{ETH Zürich} \hfill \textbf{September 2001 -- Mai 2006}\\
\vspace{1mm}%
\textsl{MSc ETH in Maschinenbau}

\section{\mysidestyle IT Fähigkeiten}\vspace{6mm}
\begin{list2}
  \item Versiert in ASP.NET Core, Entity~Framework Core und AutoMapper.
  \item Gruntkenntnisse in Angular und RxJS.
  \item Erfahren in Architektur von Microservice (REST) Diensten mittels Domain
    Driven Design.
  \item Geübt in der Erstellung von Lösungen in der Microsoft Azure Cloud;
    Infrastructure as Code mittels RedHat Ansible; CI / CD mit Jenkins.
  \item Microsoft Office, geübt im Umgang mit Power Query/Pivot und Power BI.
  \item Flie{\ss}end in C\#, .NET, WPF, MVVM, C++, C, Python, CMake, CTest,
    MATLAB, Fortran, Bash Skripten, UNIX Tools (Vim, sed, awk, grep, make,
    etc.) und {\fontfamily{cmss}\selectfont\LaTeX}.
  \item Befürworter von testgetriebenen Entwicklungsmethoden.
  \item Sehr erfahren mit den Versionskontrollsystemen Git und Subversion.
\end{list2}

\section{\mysidestyle Sprachkenntnisse}\vspace{2mm}
\begin{tabular}{@{}ll}
  \textsl{Deutsch}     & Muttersprache \\
  \textsl{Englisch}    & Flie{\ss}end in Schrift und Sprache (C2) \\
  \textsl{Französisch} & Selbständige Sprachverwendung (B2) \\
\end{tabular}

\section{\mysidestyle Publikationen}\vspace{2mm}
M.~Wild
``General Purpose PDF solution algorithm for reactive flow simulations in
OpenFOAM'', \textsl{Doctoral Thesis}, 2013

\vspace{-3mm}
H.~Xiao, Y.~Sakai, R.~Henniger, M.~Wild, P.~Jenny
``Coupling of Solvers With Non-conforming computational Domains in a Dual-Mesh
Hybrid LES/RANS Framework'', \textsl{Computers and Fluids}, 2013

\vspace{-3mm}
H.~Xiao, M.~Wild, P.~Jenny
``Preliminary evaluation and applications of a consistent hybrid LES-RANS
method'', \textsl{Notes on Numerical Fluid Mechanics and Multidisciplinary
Design}, 2012

\vspace{-3mm}
H.~Xiao, M.~Wild und P.~Jenny ``Preliminary Evaluation and
Applications of a Consistent Hybrid LES--RANS Method'', Springer Series:
\textsl{Notes on Numerical Fluid Mechanics and Multidisciplinary Design}, 2011

\vspace{-3mm}
Michael~A.~Wild, Benjamin~T.~Zoller and Patrick~Jenny
``Efficient NO Calculations in Turbulent Non-Premixed Flames Using
PDF Methods'', \textsl{Proceedings of the European Combustion Meeting}, 2009

%%%%%%%%%%%%%%%%%%%%%%%%%%%%%%%%%%%%%%%%%%%%%%%%%%%%%%%%%%%%%%%%%%%%%%%%%%%%%%%


\end{resume}
\end{document}

% vim: set expandtab shiftwidth=2 :
