\documentclass[line,11pt,a4paper]{resume}

\usepackage{graphicx}
\usepackage[pdfborder={0 0 0}]{hyperref}
\usepackage{multicol}

\newcommand{\mail}[1]{\textsl{\href{mailto:#1}{#1}}}

\addtolength{\topskip}{8mm}

\begin{document}
\name{\LARGE Michael~A.~Wild}
\begin{resume}

\begin{multicols}{2}

\section{\mysidestyle Contact\\Information}\vspace{2mm}

Leh\'{a}rstra{\ss}e 10 \\
4020 Linz \\
Austria \\
+43 660 2243450 \\
\mail{mwild@bluewin.ch}

\section{\mysidestyle Personal Information}\vspace{2mm}

\begin{tabular}{@{}ll}
\textsl{Date of birth} & 1979-12-17 \\
\textsl{Nationality}   & Swiss
\end{tabular}

\columnbreak
\vspace*{-9mm}\hfill\includegraphics[width=45mm]{mwild}

\end{multicols}

\section{\mysidestyle Education}\vspace{2mm}

\textbf{Scrum.org -- Z\"uhlke Academy} \hfill \textbf{November 2013}%
\vspace{2mm}\\\vspace{1mm}%
\textsl{Professional Scrum Master (PSM I)}%
\vspace{2mm}\\\vspace{-1mm}%
%
\textbf{Institute of Fluid Dynamics -- ETH Z\"urich} \hfill \textbf{June 2006 -- December 2012}%
\vspace{2mm}\\\vspace{1mm}%
\textsl{Dr\ sc\ ETH}%
\vspace{2mm}\\\vspace{-1mm}%
%
\textbf{University of Cambridge -- ESOL Examinations, Switzerland} \hfill \textbf{June 2007}%
\vspace{2mm}\\\vspace{1mm}%
\textsl{Certificate of Proficiency in English (CPE)}%
\vspace{2mm}\\\vspace{-1mm}%
%
\textbf{ETH Z\"urich} \hfill \textbf{September 2001 -- Mai 2006}%
\vspace{2mm}\\\vspace{1mm}%
\textsl{MSc ETH in Mechanical Engineering}%
\vspace{-3mm}\\\vspace{-1mm}%

\section{\mysidestyle Professional Experience}\vspace{2mm}

\textbf{ANDRITZ Hydro AG} \hfill \textbf{May 2013 -- Now}
\vspace{2mm}\\\vspace{1mm}%
\textsl{Development Engineer CAE Tools}\\
Currently I am the agile master and system architect for the ANDRITZ
Hydro industrial IoT solution for the monitoring and optimization of hydro
power plants. The product builds on top of the ANDRITZ METRIS platform and will
be integrated with a wide range of services, ranging from maintenance and
operation optimization, to maintenance outsourcing, remote monitoring and even
outsourcing of operation to a central control room.

Recently I was also involved in the application of machine learning
and statistical analysis for the optimization of power plant operation and
predictive maintenance. Additionally, I acted as a liaison between the engineering
departments and data analytics teams, proposing analysis methods, developing
prototypes for analysis algorithms and provided the necessary domain knowledge
to the data analysts.

Further tasks involve the development and project management of in-house
engineering tools, such as for the transients simulation of hydraulic power
plants, their hydraulic layout, design and simulation tool chains. I consult
and support software developers in matters of software architecture, choice
of programming frameworks, version control, documentation and agile project
management. I am also the patent contact for ANDRITZ Hydro to the central
patents department.

In two successful projects I was the SCRUM master and introduced agile
project management and programming practices. I was able to measurably
accelerate development progress and deliver the important features earlier
to the internal clients. The return on investment was improved considerably.

I created the architecture and design of multiple micro services which
are integrated via REST over HTTP. The micro service design allowed us to
efficiently work in geographically far distributed development teams, since the
coordination effort was reduced to the well-defined interfaces.

For the transient simulation of hydro power plants I programmed a preprocessing
tool in C\# and WPF in the Microsoft .NET environment which allows the
engineers to efficiently set up a large number of parametric studies. Strong
emphasis was placed on usability and productivity.

During a one-year period I headed an international working group with eight
members whose task it was to elaborate a strategy for the harmonization of the
high performance computing infrastructure between the various ANDRITZ Hydro
R\&D sites.

\textbf{Basler Kantonalbank} \vspace{2mm}\\\vspace{1mm}%
\textsl{Financial Engineer} \hfill \textbf{January 2013 -- April 2013}\\
I developed and maintained programs that processed data from various trading
systems and transferred it to the core banking software.

\textbf{Institute of Fluid Dynamics}, ETH Z\"urich \vspace{2mm}\\\vspace{1mm}%
\textsl{Scientific Researcher} \hfill \textbf{June 2006 -- December 2012}\\
Part of my dissertation was the creation of an MPI-parallel hybrid finite
volume / transported joint-probability-density function (JPDF) solver for the
simulation of turbulent reactive flows in C++ based on the OpenFOAM CFD
library. Within this tool I developed a novel method for the coupling of the
JPDF solver with the finite volume solver, a new consistent formulation for the
inflow and outflow boundary conditions, an algorithm for the particle number
control that preserves all moments of the JPDF and an algorithm for
asynchronous, local time stepping for the particle evolution. Furthermore I
developed pre- and post-processing tools in Python.

\textsl{Assistant System Administrator} \hfill \textbf{2007 -- 2012}\\
The infrastructure at the Institute of Fluid Dynamics comprised a small
cluster, an NFS4 file server with Kerberos authentication, a mail and a web
server, about fifty workstations and thirty computers in the student lab. I
assisted the administrator in the maintenance and operation of these systems.

\textbf{RUAG Aerospace}, Emmen \vspace{2mm}\\\vspace{1mm}%
\textsl{Internship} \hfill \textbf{June -- October, 2004}\\
I was tasked with the development of a MATLAB tool for the performance estimation
of the automotive wind tunnel.

\textbf{Institute for Mechanical Systems}, ETH Z\"urich \vspace{2mm}\\\vspace{1mm}%
\textsl{Assistant Researcher} \hfill \textbf{August 2003 -- February 2004}\\
Using MATLAB I developed a tool for the simulation of the propagation of
acoustic waves which have been induced by thermal expansion created by laser
excitation. The method is used to measure the thickness of the various layers
in chip wavers.

\pagebreak
\section{\mysidestyle Project Management}\vspace{2mm}
I have five years of experiences as SCRUM master in several projects with
relatively small, but geographically distributed development teams consisting
of internal and external programmers. For multiple projects I was
the project manager and was also responsible for the budget, reporting,
documentation, coordination and the submission of project applications.

\section{\mysidestyle IT Skills}\vspace{2mm}
I am experienced in the design of REST services and the modelling of micro
services using ,,Domain Driven Design''. I am fluent in C\#, .NET, WPF, MVVM,
C++, C, Python, CMake, CTest, MATLAB, Fortran, Bash scripting, UNIX Tools (Vim,
sed, awk, grep, GNU make, etc.) and {\fontfamily{cmss}\selectfont\LaTeX}.
I promote and consequently apply test-first development methodologies (test
driven design, TDD). I have basic skill in Java, DocBook XML+XSL, autotools and
SCons. I am very experienced with the revision control systems Git and
Subversion and have basic skills in CVS and Mercurial and have some experience
with the Qt and MPI libraries.

\section{\mysidestyle Language Skills}\vspace{2mm}
\begin{tabular}{@{}ll}
  \textsl{German}   & Native \\
  \textsl{English}  & Fluent proficiency, spoken and written (C2) \\
  \textsl{French}  & Upper intermediate (B2) \\
\end{tabular}

\section{\mysidestyle Teaching}\vspace{2mm}

\textsl{Lecturer}\\
Semi-annual course introducing about forty students to the simulation and
programming with OpenFOAM over the duration of three afternoons.

\textsl{Teaching Assistant}\\
OpenFOAM training at the Lucerne University of Applied Sciences and Arts, HSLU
(2008 and 2009). \\
Master-Lecture ,,Turbulence Modelling'' (20 -- 40 students).

\textsl{Thesis Supervision}\\
I supervised three Bachelor-, four Semester- and three Master-Projects. Amongst
others the topics were:
\begin{list2}
  \item Implementation of a stationary flamelet model for turbulent diffusion
    flames in a hybrid FV/JPDF solver in OpenFOAM.
  \item The efficient simulation of nitrogen oxide formation in turbulent
    diffusion flames.
  \item The implementation of consistent particle inflow and outflow boundary
    conditions for a hybrid FV/JPDF solver in OpenFOAM.
  \item Simulation of an oscillating cylinder in cross flow with OpenFOAM.
\end{list2}

\pagebreak
\section{\mysidestyle Publications}\vspace{2mm}
H.~Xiao, M.~Wild und P.~Jenny ``Preliminary Evaluation and
Applications of a Consistent Hybrid LES--RANS Method'', Springer Series:
\textsl{Notes on Numerical Fluid Mechanics and Multidisciplinary Design}, 2011

\vspace{-2mm}
H.~Xiao, Y.~Sakai, R.~Henniger, M.~Wild, P.~Jenny
``Coupling of Solvers With Non-conforming computational Domains in a Dual-Mesh
Hybrid LES/RANS Framework'', \textsl{Computers and Fluids}, 2013

M.~Wild
``General Purpose PDF solution algorithm for reactive flow simulations in
OpenFOAM'', \textsl{Doctoral Thesis}, 2013

\vspace{-2mm}
H.~Xiao, M.~Wild, P.~Jenny
``Preliminary evaluation and applications of a consistent hybrid LES-RANS
method'', \textsl{Notes on Numerical Fluid Mechanics and Multidisciplinary
Design}, 2012

\vspace{-2mm}
Michael~A.~Wild, Benjamin~T.~Zoller and Patrick~Jenny
``Efficient $\mathrm{NO}$ Calculations in Turbulent Non-Premixed Flames Using
PDF Methods'', \textsl{Proceedings of the European Combustion Meeting}, 2009

\vspace{-2mm}
Benjamin~T.~Zoller, Michael~A.~Wild and Patrick~Jenny
``Efficient $\mathrm{NO}$ Calculations in Reactive Flows Using PDF Methods'',
\textsl{32nd International Symposium on Combustion}, Poster presentation, 2008

%%%%%%%%%%%%%%%%%%%%%%%%%%%%%%%%%%%%%%%%%%%%%%%%%%%%%%%%%%%%%%%%%%%%%%%%%%%%%%%


\end{resume}
\end{document}

% vim: set expandtab shiftwidth=2 :
