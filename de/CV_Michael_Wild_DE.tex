\documentclass[line,11pt,a4paper]{../resume}

\usepackage{graphicx}
\usepackage[ngerman]{babel}
\usepackage[pdfborder={0 0 0}]{hyperref}
\usepackage{multicol}
\usepackage{fontawesome}

\newcommand{\mail}[1]{\textsl{\href{mailto:#1}{#1}}}

\addtolength{\topskip}{8mm}

\begin{document}
\name{\LARGE Michael~A.~Wild}
\begin{resume}
\vspace{-4mm}
\-\hspace{-12.3mm}\begin{minipage}{15cm}
\href{https://linkedin.com/in/wildmichael}{\faLinkedinSquare/in/wildmichael}\quad
\href{https://xing.com/profile/Michael_Wild4/cv}{\faXingSquare/profile/Michael\_Wild4/cv}\quad
\href{https://github.com/wildmichael}{\faGithubSquare/wildmichael}
\end{minipage}


\begin{multicols}{3}

\section{\mysidestyle Kontakt\\Daten}\vspace{0.9mm}

Leh\'{a}rstra{\ss}e 10 \\
4020 Linz \\
Austria \\
+43 660 2243450 \\
\mail{mwild@bluewin.ch}\\

\columnbreak

\section{\mysidestyle Zur Person}\vspace{2mm}

\begin{tabular}{@{}ll}
\textsl{Geburtstag} & 1979-12-17 \\
\textsl{Nationalität}   & Schweizer
\end{tabular}

\columnbreak
\vspace*{-9mm}\hfill\includegraphics[width=30mm]{../mwild}

\end{multicols}

\section{\mysidestyle Ausbildung}\vspace{2mm}

\textbf{Scrum.org -- Zühlke Academy} \hfill \textbf{November 2013}%
\vspace{2mm}\\\vspace{1mm}%
\textsl{Professional Scrum Master (PSM I)}%
\vspace{2mm}\\\vspace{-1mm}%
%
\textbf{Institut für Fluid Dynamik -- ETH Zürich} \hfill \textbf{Juni 2006 -- Dezember 2012}%
\vspace{2mm}\\\vspace{1mm}%
\textsl{Dr\ sc\ ETH}%
\vspace{2mm}\\\vspace{-1mm}%
%
\textbf{University of Cambridge -- ESOL Examinations, Switzerland} \hfill \textbf{Juni 2007}%
\vspace{2mm}\\\vspace{1mm}%
\textsl{Certificate of Proficiency in English (CPE)}%
\vspace{2mm}\\\vspace{-1mm}%
%
\textbf{ETH Zürich} \hfill \textbf{September 2001 -- Mai 2006}%
\vspace{2mm}\\\vspace{1mm}%
\textsl{MSc ETH in Maschinenbau}%
\vspace{-3mm}\\\vspace{-1mm}%

\section{\mysidestyle Berufserfahrung}\vspace{2mm}

\textbf{ANDRITZ Hydro} \hfill \textbf{Jan 2019 -- Heute}
\vspace{2mm}\\\vspace{1mm}%
\textsl{Teamleiter Digitalisierung}
\begin{list2}
  \item Entwicklung und Wartung verschiedener Softwarepackete, die von der F\&E,
    dem technischen Verkauf und der Konstruktionsabteilung verwendet werden.
  \item Erstellen einer Datenbank für die parametrische CAD-Konstruktion
    basierend auf den internen Richtlinien. Integration in bestehende Systeme
    und Erstellung von Schnittstellen für Berechnungswerkzeuge.
  \item Enge Einbindung mehrer bestehender Lösungen, Erstellen von digitalen
    Datenflüssen um die manuelle Eingabe von Daten zu reduzieren.
\end{list2}


\textbf{ANDRITZ Hydro} \hfill \textbf{Mai 2013 -- Dez 2018}
\vspace{2mm}\\\vspace{1mm}%
\textsl{Entwicklungsingenieur CAE Werkzeuge}\\
\begin{list2}
  \item Agile Master und Systemarchitekt für die Industrial IoT Lösung zur
    Überwachung und Optimierung von Wasserkraftanlagen basierend auf der
    ANDRITZ METRIS.

  \item Anwendung von Machine Learning und statistischer Analyse für die
    Betriebsoptimierung von Wasserkraftanlagen und vorausschauender Wartung.
    Verbindung zwischen den Entwicklungsabteilungen und dem Datenanalyseteam.
    Einbringung des notwendigen Domänenwissens, Entwicklung von
    Prototyp-Algorithmen.

  \item Entwickler und Projektleiter für von ANDRITZ selbst entwickelte
    Berechnungtools, z.B.\ für die Simulation des transienten Verhaltens von
    Wasserkraftwerken, deren Dimensionierung im technischen Verkauf, oder der
    Designwerkzeuge und Simulationsketten für die Turbinenentwicklung.

  \item Kontakt der Turbinenentwicklung zur Patentabteilung.

  \item Scrum Master in zwei erfolgreichen Projekten. Der
    Entwicklungsfortschritt wurde me{\ss}bar beschleunigt, und dabei dank
    Priorisierung die wichtigen Funktionalitäten den internen Kunden früher zur
    Verfügung gestellt. Die Rentabiliät der Investition wurde dadurch deutlich
    verbessert.

  \item Architekt und Designer mehrerer Microservices, die mittels
    REST-Schnittstellen über HTTP kommunizieren. Die Dienste wurden von
    geographisch weit verteilten Teams entwickelt.

  \item Programmierer eines Werkzeugs zur Vorbereitung von Simulationen des
    transienten Verhaltens von Wasserkraftwerken, geschrieben in C\# und WPF.

  \item Leiter einer achtköpfigen internationalen Arbeitsgruppe zur Erstellung
    einer Harmonisierungstrategie für die Hochleistungsrechner zwischen den
    verschiedenen ANDRITZ Hydro F\&E Standorten.
\end{list2}

\textbf{Basler Kantonalbank} \vspace{2mm}\\\vspace{1mm}%
\textsl{Financial Engineer} \hfill \textbf{Januar 2013 -- April 2013}\\
\begin{list2}
  \item Entwicklung und Wartung der Programme zur Verarbeitung und
    Zusammenführung von Daten der unterschiedlichen Handelssystem in die
    Kernbankensoftware.
\end{list2}

\textbf{Institut für Fluid Dynamik}, ETH Zürich \vspace{2mm}\\\vspace{1mm}%
\textsl{Wissenschaftlicher Mitarbeiter} \hfill \textbf{Juni 2006 -- Dezember 2012}\\
\begin{list2}
  \item Entwickler eines MPI-parallen Lösers mittels einem hybriden Ansatz aus
    der traditionellen Finite-Volumen-Methode und der Transportgleichung für
    die Verbundswahrscheinlichkeitsdichtefunktion mittels einer
    Partikel-Monte-Carlo Methode. Die Software basierte auf der Open-Source
    Bibliothek OpenFOAM.

  \item Erfinder eine neuen Methode zur Kopplung der hybriden Gleichungen,
    einer konsistenten Formulierung für die Ein- und Auslassrandbedinugnen,
    eines neuen Algorithmus für die Kontrolle der Partikelanzahl, welcher alle
    Momente der Verbundswahrscheinlichkeitsdichtefunktion erhält, und eines
    neuen Verfahrens für den asynchronen Transport der Partikel mittels lokaler
    Zeitschrittweite.

  \item Erstellung einer Vielzahl an Vor- und Nachbereitungs-Werkzeugen in
    Python.
\end{list2}

\section{\mysidestyle Projektleitung}\vspace{6mm}
\begin{list2}
  \item Sechs Jahre Erfahrung als Scrum Master in mehreren Projekten mit
    kleinen, aber geographisch weit verteilten Entwicklungsteams. Diese
    bestanden sowohl aus internen und externen Programmierern.

  \item Projektleiter in mehreren Projekten mit Verantwortung für die
    Budgeteinhaltung, die Berichterstattung und Dokumentation.
\end{list2}

\pagebreak
\section{\mysidestyle IT Fähigkeiten}\vspace{6mm}
\begin{list2}
  \item Erfahren im Design von REST Diensten und der Modellierung von
    Microservice-Architekturen mittels ,,Domain Driven Design''.
  \item Flie{\ss}end in C\#, .NET, WPF, MVVM, C++, C, Python, CMake, CTest,
    MATLAB, Fortran, Bash Skripten, UNIX Tools (Vim, sed, awk, grep, make,
    etc.) und {\fontfamily{cmss}\selectfont\LaTeX}.
  \item Geübt in der Anwendung von Ansible in Verbindung mit Microsoft Azure.
  \item Grundlegende Kenntnisse in Java, DocBoox XML+XSL, autotools und SCons.
  \item Befürworter von testgetriebenen Entwicklungsmethoden.
  \item Sehr erfahren mit den Versionskontrollsystemen Git und Subversion.
  \item Arbeitserfahrung mit den Qt- und MPI-Bibliotheken.
\end{list2}

\section{\mysidestyle Sprachkentnisse}\vspace{2mm}
\begin{tabular}{@{}ll}
  \textsl{Deutsch}   & Muttersprache \\
  \textsl{Englisch}  & Flie{\ss}end in Schrift und Sprache (C2) \\
  \textsl{Französisch}  & Selbständige Sprachverwendung (B2) \\
\end{tabular}

\section{\mysidestyle Unterricht}\vspace{2mm}

\textsl{Dozent}\\
Halbjährlich stattfindender Kurs, der jeweils etwa vierzig Studenten während
drei Nachmittagen in die Strömungssimulation mittels OpenFOAM einführte.

\textsl{Unterrichtsassistent}
\begin{list2}
  \item OpenFOAM-Training an der Fachhochschule Luzern, HSLU. (2008 and 2009).
  \item Master-Vorlesung ,,Turbulenzmodellierung'' (20 -- 40 Studenten).
\end{list2}

\textsl{Betreuung von Studentenarbeiten}\\
Betreuung von drei Bachelor-, vier Semester- und drei Master-Arbeiten.

\section{\mysidestyle Publikationen}\vspace{2mm}
M.~Wild
``General Purpose PDF solution algorithm for reactive flow simulations in
OpenFOAM'', \textsl{Doctoral Thesis}, 2013

\vspace{-2mm}
H.~Xiao, Y.~Sakai, R.~Henniger, M.~Wild, P.~Jenny
``Coupling of Solvers With Non-conforming computational Domains in a Dual-Mesh
Hybrid LES/RANS Framework'', \textsl{Computers and Fluids}, 2013

\vspace{-2mm}
H.~Xiao, M.~Wild, P.~Jenny
``Preliminary evaluation and applications of a consistent hybrid LES-RANS
method'', \textsl{Notes on Numerical Fluid Mechanics and Multidisciplinary
Design}, 2012

\vspace{-2mm}
H.~Xiao, M.~Wild und P.~Jenny ``Preliminary Evaluation and
Applications of a Consistent Hybrid LES--RANS Method'', Springer Series:
\textsl{Notes on Numerical Fluid Mechanics and Multidisciplinary Design}, 2011

\vspace{-2mm}
Michael~A.~Wild, Benjamin~T.~Zoller and Patrick~Jenny
``Efficient $\mathrm{NO}$ Calculations in Turbulent Non-Premixed Flames Using
PDF Methods'', \textsl{Proceedings of the European Combustion Meeting}, 2009

\vspace{-2mm}
Benjamin~T.~Zoller, Michael~A.~Wild and Patrick~Jenny
``Efficient $\mathrm{NO}$ Calculations in Reactive Flows Using PDF Methods'',
\textsl{32nd International Symposium on Combustion}, Poster presentation, 2008

%%%%%%%%%%%%%%%%%%%%%%%%%%%%%%%%%%%%%%%%%%%%%%%%%%%%%%%%%%%%%%%%%%%%%%%%%%%%%%%


\end{resume}
\end{document}

% vim: set expandtab shiftwidth=2 :
