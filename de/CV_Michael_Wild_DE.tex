\documentclass[line,11pt,a4paper]{../resume}

\usepackage{graphicx}
\usepackage[ngerman]{babel}
\usepackage[pdfborder={0 0 0}]{hyperref}
\usepackage{multicol}
\usepackage{fontawesome}

\newcommand{\mail}[1]{\textsl{\href{mailto:#1}{#1}}}

\addtolength{\topskip}{8mm}

\begin{document}
\name{\LARGE Michael~A.~Wild}
\begin{resume}
\vspace{-4mm}
\-\hspace{-12.3mm}\begin{minipage}{15cm}
\href{https://linkedin.com/in/wildmichael}{\faLinkedinSquare/in/wildmichael}\quad
\href{https://xing.com/profile/Michael_Wild4/cv}{\faXingSquare/profile/Michael\_Wild4/cv}\quad
\href{https://github.com/wildmichael}{\faGithubSquare/wildmichael}
\end{minipage}


\begin{multicols}{3}

\section{\mysidestyle Kontakt\\Daten}\vspace{0.9mm}

Leh\'{a}rstra{\ss}e 10 \\
4020 Linz \\
Austria \\
+43 660 2243450 \\
\mail{mwild@bluewin.ch}\\

\columnbreak

\section{\mysidestyle Zur Person}\vspace{2mm}

\begin{tabular}{@{}ll}
\textsl{Geburtstag} & 1979-12-17 \\
\textsl{Nationalität}   & Schweizer
\end{tabular}

\columnbreak
\vspace*{-9mm}\hfill\includegraphics[width=30mm]{../mwild}

\end{multicols}

\section{\mysidestyle Berufserfahrung}\vspace{2mm}

\textbf{ANDRITZ Hydro}\\\vspace{1mm}%
\textsl{Teamleiter Digitalisierung} \hfill \textbf{Januar 2019 -- Heute}\\
\vspace{-4mm}%
\begin{list2}
  \item Entwicklung und Wartung verschiedener Softwarepackete, die von der F\&E,
    dem technischen Verkauf und der Konstruktionsabteilung verwendet werden.
  \item Erstellen einer Datenbank für die parametrische CAD-Konstruktion
    basierend auf den internen Richtlinien. Integration in bestehende Systeme
    und Erstellung von Schnittstellen für Berechnungswerkzeuge.
  \item Enge Einbindung mehrer bestehender Lösungen, Erstellen von digitalen
    Datenflüssen um die manuelle Eingabe von Daten zu reduzieren.
\end{list2}

\textsl{Entwicklungsingenieur CAE Werkzeuge} \hfill \textbf{Mai 2013 -- Dezember 2018}\\
\vspace{-4mm}%
\begin{list2}
  \item Agile Master und Systemarchitekt für die Industrial IoT Lösung METRIS
    DiOMera zur Überwachung und Optimierung von Wasserkraftanlagen; basierend
    auf der ANDRITZ METRIS Platform.

  \item Anwendung von Machine Learning und statistischer Analyse für die
    Betriebsoptimierung von Wasserkraftanlagen und vorausschauende Wartung.
    Vorschlagen von Analysemethoden, Einbringung des notwendigen
    Domänenwissens, Entwicklung von Prototyp-Algorithmen.

  \item Leiter einer internationalen Arbeitsgruppe zur Erstellung
    einer Harmonisierungstrategie für die Hochleistungsrechner zwischen den
    verschiedenen ANDRITZ Hydro F\&E Standorten.

  \item Unter anderem Entwickler und Projektleiter von internen
    Engineering-Tools; Kontaktperson zur Patentabteilung; SCRUM Master in
    mehreren erfolgreichen Projekten; Architekt und Designer mehrerer
    Microservices.
\end{list2}

\textbf{Basler Kantonalbank}\\\vspace{1mm}%
\textsl{Financial Engineer} \hfill \textbf{Januar 2013 -- April 2013}\\
\vspace{-4mm}%
\begin{list2}
  \item Entwicklung und Wartung der Programme zur Verarbeitung und
    Zusammenführung von Daten der unterschiedlichen Handelssystem in die
    Kernbankensoftware.
\end{list2}

\textbf{Institut für Fluid Dynamik}, ETH Zürich\\\vspace{1mm}%
\textsl{Wissenschaftlicher Mitarbeiter} \hfill \textbf{Juni 2006 -- Dezember 2012}\\
\vspace{-4mm}%
\begin{list2}
  \item Entwickler eines MPI-parallen Lösers mittels einem hybriden Ansatz aus
    der traditionellen Finite-Volumen-Methode und der Transportgleichung für
    die Verbundswahrscheinlichkeitsdichtefunktion mittels einer
    Partikel-Monte-Carlo Methode. Die Software basierte auf der Open-Source
    Bibliothek OpenFOAM.

  \item Erfinder mehrer neueartiger Algorithmen, welche die Stabilität und
    Effizenz der hybriden FV-JPDF Simulationsmethode verbesserten.
\end{list2}

\section{\mysidestyle Ausbildung}\vspace{2mm}

\textbf{Scrum.org -- Zühlke Academy} \hfill \textbf{November 2013}\\
\vspace{1mm}%
\textsl{Professional Scrum Master (PSM I)}\\
\vspace{-1mm}%
%
\textbf{Institut für Fluid Dynamik -- ETH Zürich} \hfill \textbf{Juni 2006 -- Dezember 2012}\\
\vspace{1mm}%
\textsl{Dr\ sc\ ETH}\\
\vspace{-1mm}%
%
\textbf{University of Cambridge -- ESOL Examinations, Switzerland} \hfill \textbf{Juni 2007}\\
\vspace{1mm}%
\textsl{Certificate of Proficiency in English (CPE)}\\
\vspace{-1mm}%
%
\textbf{ETH Zürich} \hfill \textbf{September 2001 -- Mai 2006}\\
\vspace{1mm}%
\textsl{MSc ETH in Maschinenbau}

\section{\mysidestyle Projektleitung}\vspace{6mm}
\begin{list2}
  \item Sechs Jahre Erfahrung als SCRUM Master in mehreren Projekten mit
    kleinen, aber geographisch weit verteilten Entwicklungsteams. Diese
    bestanden sowohl aus internen als auch aus externen Programmierern.

  \item Projektleiter in mehreren Projekten mit Verantwortung für die
    Budgeteinhaltung, die Berichterstattung und Dokumentation.
\end{list2}

\pagebreak
\section{\mysidestyle IT Fähigkeiten}\vspace{6mm}
\begin{list2}
  \item Erfahren im Design von REST Diensten und der Modellierung von
    Microservice-Architekturen mittels Domain Driven Design.
  \item Geübt in der Erstellung von Lösungen in der Microsoft Azure Cloud;
    Definieren der Infrastruktur mittels RedHat Ansible; Continuous Delivery
    und Integration mittels Jenkins.
  \item Microsoft Office, geübt im Umgang mit Power Query/Pivot und Power BI.
  \item Flie{\ss}end in C\#, .NET, WPF, MVVM, C++, C, Python, CMake, CTest,
    MATLAB, Fortran, Bash Skripten, UNIX Tools (Vim, sed, awk, grep, make,
    etc.) und {\fontfamily{cmss}\selectfont\LaTeX}.
  \item Grundlegende Kenntnisse in Java.
  \item Befürworter von testgetriebenen Entwicklungsmethoden.
  \item Sehr erfahren mit den Versionskontrollsystemen Git und Subversion.
\end{list2}

\section{\mysidestyle Sprachkentnisse}\vspace{2mm}
\begin{tabular}{@{}ll}
  \textsl{Deutsch}   & Muttersprache \\
  \textsl{Englisch}  & Flie{\ss}end in Schrift und Sprache (C2) \\
  \textsl{Französisch}  & Selbständige Sprachverwendung (B2) \\
\end{tabular}

\section{\mysidestyle Unterricht}\vspace{2mm}

\textsl{Dozent}\\
Halbjährlich stattfindender Kurs, der jeweils etwa vierzig Studenten während
drei Nachmittagen in die Strömungssimulation mittels OpenFOAM einführte.

\textsl{Unterrichtsassistent}
\begin{list2}
  \item OpenFOAM-Training an der Fachhochschule Luzern, HSLU. (2008 and 2009).
  \item Master-Vorlesung ,,Turbulenzmodellierung'' (20 -- 40 Studenten).
\end{list2}

\textsl{Betreuung von Studentenarbeiten}\\
Betreuung von drei Bachelor-, vier Semester- und drei Master-Arbeiten.

\section{\mysidestyle Publikationen}\vspace{2mm}
M.~Wild
``General Purpose PDF solution algorithm for reactive flow simulations in
OpenFOAM'', \textsl{Doctoral Thesis}, 2013

\vspace{-2mm}
H.~Xiao, Y.~Sakai, R.~Henniger, M.~Wild, P.~Jenny
``Coupling of Solvers With Non-conforming computational Domains in a Dual-Mesh
Hybrid LES/RANS Framework'', \textsl{Computers and Fluids}, 2013

\vspace{-2mm}
H.~Xiao, M.~Wild, P.~Jenny
``Preliminary evaluation and applications of a consistent hybrid LES-RANS
method'', \textsl{Notes on Numerical Fluid Mechanics and Multidisciplinary
Design}, 2012

\vspace{-2mm}
H.~Xiao, M.~Wild und P.~Jenny ``Preliminary Evaluation and
Applications of a Consistent Hybrid LES--RANS Method'', Springer Series:
\textsl{Notes on Numerical Fluid Mechanics and Multidisciplinary Design}, 2011

\vspace{-2mm}
Michael~A.~Wild, Benjamin~T.~Zoller and Patrick~Jenny
``Efficient $\mathrm{NO}$ Calculations in Turbulent Non-Premixed Flames Using
PDF Methods'', \textsl{Proceedings of the European Combustion Meeting}, 2009

\vspace{-2mm}
Benjamin~T.~Zoller, Michael~A.~Wild and Patrick~Jenny
``Efficient $\mathrm{NO}$ Calculations in Reactive Flows Using PDF Methods'',
\textsl{32nd International Symposium on Combustion}, Poster presentation, 2008

%%%%%%%%%%%%%%%%%%%%%%%%%%%%%%%%%%%%%%%%%%%%%%%%%%%%%%%%%%%%%%%%%%%%%%%%%%%%%%%


\end{resume}
\end{document}

% vim: set expandtab shiftwidth=2 :
