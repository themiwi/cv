\documentclass[line,11pt,a4paper]{../resume}

\usepackage{graphicx}
\usepackage[ngerman]{babel}
\usepackage[pdfborder={0 0 0}]{hyperref}
\usepackage{multicol}

\newcommand{\mail}[1]{\textsl{\href{mailto:#1}{#1}}}

\addtolength{\topskip}{8mm}

\begin{document}
\name{\LARGE Michael~A.~Wild}
\begin{resume}

\begin{multicols}{2}

\section{\mysidestyle Kontakt\\Daten}\vspace{2mm}

Leh\'{a}rstra{\ss}e 10 \\
4020 Linz \\
Austria \\
+43 660 2243450 \\
\mail{mwild@bluewin.ch}

\section{\mysidestyle Zur Person}\vspace{2mm}

\begin{tabular}{@{}ll}
\textsl{Geburtstag} & 1979-12-17 \\
\textsl{Nationalität}   & Schweizer
\end{tabular}

\columnbreak
\vspace*{-9mm}\hfill\includegraphics[width=45mm]{../mwild}

\end{multicols}

\section{\mysidestyle Ausbildung}\vspace{2mm}

\textbf{Scrum.org -- Zühlke Academy} \hfill \textbf{November 2013}%
\vspace{2mm}\\\vspace{1mm}%
\textsl{Professional Scrum Master (PSM I)}%
\vspace{2mm}\\\vspace{-1mm}%
%
\textbf{Institut für Fluid Dynamik -- ETH Zürich} \hfill \textbf{Juni 2006 -- Dezember 2012}%
\vspace{2mm}\\\vspace{1mm}%
\textsl{Dr\ sc\ ETH}%
\vspace{2mm}\\\vspace{-1mm}%
%
\textbf{University of Cambridge -- ESOL Examinations, Switzerland} \hfill \textbf{Juni 2007}%
\vspace{2mm}\\\vspace{1mm}%
\textsl{Certificate of Proficiency in English (CPE)}%
\vspace{2mm}\\\vspace{-1mm}%
%
\textbf{ETH Zürich} \hfill \textbf{September 2001 -- Mai 2006}%
\vspace{2mm}\\\vspace{1mm}%
\textsl{MSc ETH in Maschinenbau}%
\vspace{-3mm}\\\vspace{-1mm}%

\section{\mysidestyle Berufserfahrung}\vspace{2mm}

\textbf{ANDRITZ Hydro AG} \hfill \textbf{Mai 2013 -- Heute}
\vspace{2mm}\\\vspace{1mm}%
\textsl{Entwicklungsingenieur CAE Werkzeuge}\\
In meiner jetzigen Funktion bin ich der Agile Master und Systemarchitekt für
die Industrial IoT Lösung zur Überwachung und Optimierung von
Wasserkraftanlagen. Das Produkt baut auf der ANDRITZ METRIS Platform auf und
wird in eine breites Angebot von Dienstleistungen eingebettet, von der
Wartungs- und Betriebsoptimierung, über Wartuns-Outsourcing bis zum
Fernbetrieb aus einer Zentralwarte durch ANDRITZ.

Aktuell bin ich auch in der Anwendung von Machine Learning und statistischer
Analyse für die Betriebsoptimierung von Wasserkraftanlagen und
vorausschauender Wartung involviert. Zusätzlich war ich die Verbindung
zwischen den Entwicklungsabteilungen und dem Datenanalyseteam. Ich habe dabei
das notwendige Domänenwissen eingebracht, Algorithmen vorgeschlagen und
Prototypen entwickelt.

Weitere Aufgaben bestehen darin, von ANDRITZ selbst entwickelte
Berechnungtools zu entwickeln und warten, z.B. für die Simulation des
transienten Verhaltens von Wasserkraftwerken, deren Dimensionierung im
technischen Verkauf, oder der Designwerkzeuge und Simulationsketten für die
Turbinenentwicklung. Ich berate und unterstütze andere Softwareentwickler zu
Fragen der Architektur, Auswahl und Verwendung von Programmierumgebungen und
Bibliotheken, Versionskontrolle, Dokumentation und agilem Projektmanagement.
Ich bin auch der Kontakt der Turbinenentwicklung zur Patentabteilung.

In zwei erfolgreichen Projekten war ich der Scrum Master, und habe selbst das
agile Projektmanagement und die agilen Entwicklungspraktiken eingeführt.
Den Entwicklungsfortschritt konnte ich deutlich beschleunigen, und dabei dank
Priorisierung die wichtigen Funktionalitäten den internen Kunden früher zur
Verfügung stellen. Die Rentabiliät der Investition wurde dadurch deutlich
verbessert.

Für mehrere Microservices, die mittel REST-Schnittstellen über HTTP
kommunizieren, habe ich die Architektur entwickelt. Dank dieser Wahl waren die
Schnittstellen klar definiert und die geographisch weit verteilten Entwickler
konnten so effizienter und unabhängiger voneinander arbeiten.

Für die Transientensimulation von Wasserkraftwerken habe ich ein Werkzeug
zur Vorbereitung der Simulationen in C\# und WPF in der Microsoft .NET
Umgebung entwickelt. Damit können die Ingenieure rasch und produktiv eine
gro{\ss}e Anzahl von Simulationen für parametrische Studien erstellen, und
diese dann auch in den Projektbericht übernehmen. Der Hauptfokus lag dabei
auf Benutzerfreundlichkeit und Produktivität.

Für ein Jahr leitete ich eine internationale Arbeitsgruppe, deren Aufgabe
es war, eine Strategie zur Harmonisierung der
Hochleistungsrechner-Infrastruktur zwischen mehreren ANDRITZ Hydro Standorten
zu erarbeiten.

\textbf{Basler Kantonalbank} \vspace{2mm}\\\vspace{1mm}%
\textsl{Financial Engineer} \hfill \textbf{Januar 2013 -- April 2013}\\
Ich entwickelte und wartete Programme zur Verarbeitung und
Zusammenfürung von Daten aus unterschiedlichen Handelsplatformen in die
Kernbankensoftware.

\textbf{Institut für Fluid Dynamik}, ETH Zürich \vspace{2mm}\\\vspace{1mm}%
\textsl{Wissenschaftlicher Mitarbeiter} \hfill \textbf{Juni 2006 -- Dezember 2012}\\
Teil meiner Dissertation war das Entwickeln eines MPI-parallen Lösers mittels
einem hybriden Ansatz aus der traditionellen Finite-Volumen-Methode und der
Transportgleichung für die Verbundswahrscheinlichkeitsdichtefunktion mittels
einer Partikel-Monte-Carlo Methode. Die Software basierte auf der Open-Source
Bibiliothek OpenFOAM und diente der Simulation reaktiver, turbulenter
Strömungen wie sie z.B. in Gasturbinen auftreten. In dieser Software
entwickelte ich eine neue Methode zur Kopplung der hybriden Gleichungen, eine
konsistente Formulierung für die Ein- und Auslassrandbedinugnen, ein neues
Algorithmus für die Kontrolle der Partikelanzahl, welche alle Momente der
Verbundswahrscheinlichkeitsdichtefunktion erhält, und einen neuen Verfahren
für den asynchronen Transport der Partikel mittels lokaler Zeitschrittweite.
Dazu schrieb ich eine Vielzahl an Vor- und Nachbereitungs-Werkzeugen in
Python.

\textsl{Assistierender Systemadministrator} \hfill \textbf{2007 -- 2012}\\
Die Infrastruktur des Instituts für Fluid Dynamik umfasste einen kleinen
Rechencluster, ein NFS4 Dateiserver mit Kerberos-Authentifizierung, einen
Mail- und Webserver, ungefähr fünfzig Workstations und dreissig Rechner
im Studentenraum. Den Systemadministrator unterstützte ich bei der
Wartung und dem Betrieb der Systeme und machte die Urlaubsvertretung.

\textbf{RUAG Aerospace}, Emmen \vspace{2mm}\\\vspace{1mm}%
\textsl{Praktikum} \hfill \textbf{Juni -- Oktober, 2004}\\
Ich wurde mit der Entwicklung eines MATLAB-Werkzuegs zur Leistungsschätzung
des Automobilwindkanals beauftragt.

\textbf{Institut für Mechanische Systeme}, ETH Zürich \vspace{2mm}\\\vspace{1mm}%
\textsl{Hilfsassistent} \hfill \textbf{August 2003 -- Februar 2004}\\
Mit MATLAB schrieb ich ein Simulationsprogramm für die Ausbreitung von
durch Erwärmung mittels Laser ausgelösten Schallwellen. Dieses wurde zur
Entwicklung einer neuen Methode zur Schichtdickenmessung von Chipwavern
verwendet.

\section{\mysidestyle Projektleitung}\vspace{2mm}
Ich habe fünf Jahre Erfahrung als Scrum Master in mehreren Projekten mit
kleinen, aber geographisch weit verteilten Entwicklungsteams. Diese bestanden
sowohl aus internen und externen Programmierern. Für mehrere Projekte war ich
der Projektleiter und war für die Budgeteinahltung, die Berichterstattung und
Dokumentation verantwortlich.

\section{\mysidestyle IT Fähigkeiten}\vspace{2mm}
Ich habe Erfahrung im Design von REST Diensten und der Modellierung von
Microservice-Architekturen mittels ,,Domain Driven Design''. Ich beherrsche
C\#, .NET, WPF, MVVM, C++, C, Python, CMake, CTest, MATLAB, Fortran,
Bash-Skripten, UNIX Werkzeuge (Vim, sed, awk, grep, make, etc.) und
{\fontfamily{cmss}\selectfont\LaTeX}. Ich bin konsequent in der Anwendung von
test-getriebenen Methoden. In Java, DocBook XML+XSL, GNU autotools und
Subversion habe ich grundlegende Kenntnisse. Auch habe ich Erfahrung mit den
Qt- und MPI-Bibliotheken .

\section{\mysidestyle Sprachkentnisse}\vspace{2mm}
\begin{tabular}{@{}ll}
  \textsl{Deutsch}   & Muttersprache \\
  \textsl{English}  & Flie{\ss}end in Schrift und Sprache (C2) \\
  \textsl{Französisch}  & Selbständige Sprachverwendung (B2) \\
\end{tabular}

\section{\mysidestyle Unterricht}\vspace{2mm}

\textsl{Dozent}\\
Ich leitete einen halbjährlich stattfindenden Kurs, der jeweils etwa vierzig
Studenten während dreier Nachmittage in die Strömungssimulation mittels
OpenFOAM einführte.

\textsl{Unterichtsassistent}\\
OpenFOAM-Training an der Fachhochschule Luzern, HSLU. (2008 and 2009). \\
Master-Vorlesung ,,Turbulenzmodellierung'' (20 -- 40 Studenten).

\textsl{Betreuung von Studentenarbeiten}\\
Ich habe drei Bachelor-, vier Semester- und drei Master-Arbeiten betreut. Unter
anderem waren die Themen:
\begin{list2}
  \item Implementation eines stationären Flamelet-Modells für turbulente
    Diffusionsflammen in einem hybriden FV/JPDF Lösers mittels OpenFOAM.
  \item Die effiziente Simulation von Stickoxidbildung in turbulenten
    Diffusionsflammen.
  \item Die Implementation konsistenter Ein- und Auslassrandbedingungen für
    die Partikel in einem hybriden FV/JPDF Löser mittels OpenFOAM.
  \item Simulation eines oszillierenden Zylinders in Querströmung mit
    OpenFOAM.
\end{list2}

\pagebreak
\section{\mysidestyle Publikationen}\vspace{2mm}
H.~Xiao, M.~Wild und P.~Jenny ``Preliminary Evaluation and
Applications of a Consistent Hybrid LES--RANS Method'', Springer Series:
\textsl{Notes on Numerical Fluid Mechanics and Multidisciplinary Design}, 2011

\vspace{-2mm}
H.~Xiao, Y.~Sakai, R.~Henniger, M.~Wild, P.~Jenny
``Coupling of Solvers With Non-conforming computational Domains in a Dual-Mesh
Hybrid LES/RANS Framework'', \textsl{Computers and Fluids}, 2013

M.~Wild
``General Purpose PDF solution algorithm for reactive flow simulations in
OpenFOAM'', \textsl{Doctoral Thesis}, 2013

\vspace{-2mm}
H.~Xiao, M.~Wild, P.~Jenny
``Preliminary evaluation and applications of a consistent hybrid LES-RANS
method'', \textsl{Notes on Numerical Fluid Mechanics and Multidisciplinary
Design}, 2012

\vspace{-2mm}
Michael~A.~Wild, Benjamin~T.~Zoller and Patrick~Jenny
``Efficient $\mathrm{NO}$ Calculations in Turbulent Non-Premixed Flames Using
PDF Methods'', \textsl{Proceedings of the European Combustion Meeting}, 2009

\vspace{-2mm}
Benjamin~T.~Zoller, Michael~A.~Wild and Patrick~Jenny
``Efficient $\mathrm{NO}$ Calculations in Reactive Flows Using PDF Methods'',
\textsl{32nd International Symposium on Combustion}, Poster presentation, 2008

%%%%%%%%%%%%%%%%%%%%%%%%%%%%%%%%%%%%%%%%%%%%%%%%%%%%%%%%%%%%%%%%%%%%%%%%%%%%%%%


\end{resume}
\end{document}

% vim: set expandtab shiftwidth=2 :
